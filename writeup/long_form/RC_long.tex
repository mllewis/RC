\documentclass[man]{apa2}
\usepackage{pslatex}
\usepackage{amssymb}
\usepackage{graphicx}
\usepackage{color}
\usepackage{covington}
\usepackage[usenames,dvipsnames]{xcolor}

\title{Young children's developing sensitivity to discourse continuity as a cue for inferring reference}

\twoauthors{Molly L. Lewis}{Michael C. Frank}
\twoaffiliations{Department of Psychology, Stanford University}{Department of Psychology, Stanford University}


\abstract{

~\\

Keywords: xx}

\shorttitle{Discourse as a Cue for Inferring Reference}
\rightheader{Discourse as a Cue for Inferring Reference}

\acknowledgements{Special thanks to the staff and families at the Children's Discovery Museum of San Jose. This work supported by a John Merck Scholars Fellowship to MCF. An earlier version of this work was presented to the Cognitive Science Society in \citeA{horowitz2013}.

~\\

\noindent Address all correspondence to Molly L. Lewis, Stanford University, Department of Psychology, Jordan Hall, 450 Serra Mall (Bldg. 420), Stanford, CA, 94305. Phone: 650-721-9270. E-mail: \texttt{mll@stanford.edu}}

\begin{document}

\maketitle                            


\section{Introduction}

\subsection{Discourse Structure and Referent Selection}

\section{Experiment 1A}


\begin{figure} 
  \begin{center} 
    \includegraphics[width=6in]{figures/continuity_demo_all_trials.png} 
    \caption{\label{fig:demo} Schematic order of events for First Toy and Second Toy trials across Experiments. In Experiment 1A (\emph{Embedded} trials), the experimenter made eye contact without other gaze cues and introduced the naming event between two descriptions of a single toy.  In Experiment 1B (\emph{After} trials), events were identical except that the experimenter introduced the naming event after two descriptions of a toy.} 
  \end{center} 
\end{figure}	

\section{Methods}


\subsection{Participants}

\subsection{Stimuli}



\subsection{Procedure}



\subsection{Results}

Despite the lack of coincident social cues to disambiguate the label, older children and adults were both overall much more likely to select the toy whose descriptions surrounded the naming event, suggesting that they can recognize and make inferences from discourse continuity.  Figure \ref{fig:res5}, left, illustrates the proportion of children and adults selecting the second toy as the referent of the label across trial types (whether the label was introduced with the First Toy or the Second Toy). Three-year-olds' responses appeared to be sensitive to naming condition, and five-year-olds' ability to use topic continuity in this paradigm was comparable to adult performance.


\bibliographystyle{apacite2}
\bibliography{biblibrary}

\newpage
\theappendix 

\section{Appendix A: Materials}




\end{document}
